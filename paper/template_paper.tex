Biophysical journal 


%Be sure to follow:
%\url{http://www.ccbb.pitt.edu/Faculty/zuckerman/ScienceWritingChecklist.pdf}
 ``Multi-scale simulations of diffusion-influenced reactions''

\subsection{Abstract}
%\lbi
%\item what is the problem 
%\item What was done
%\item Key findings
%\item Implications 
%\lei

Kekenes-Huskey, PM; Eun, Changsun and JA McCammon

Biochemical reaction networks connect molecular signaling events with cellular function. 
A subset of these reactions are strongly influenced by diffusion "barriers" arising from impenetrable cellular structures and macromolecules, as well as interactions between the reaction substrate, its environment and its target enzyme.
For these diffusion-influenced reactions, the spatial organization of diffusion barriers, specific and non-specific binding kinetics, and molecular composition may determine the timescale and amplitude of substrate signals. 
To this end we have developed simulation tools to determine how densely-packed diffusion obstacles, like proteins, and cellular structures give rise to  ’compartments’ within which substrate signals are decoupled, and how this decoupling depends on obstacle distribution and composition.
A key advance of this approach is the use of homogenization theory to coarse-grain hindrance of diffusion due to substrate/obstacle interactions.




\subsection{Introduction}
%\lbi
%\item INSERT MAJOR FIGURE FOR ENTIRE PAPER RIGHT HERE - write paper around this image 
%\item General background and specifics on system of interest
%\item Prior work that has been done
%\item explanaton of what was done in paper 
%\item key findings/spec aims
%\lei

Coupled, diffusion-influenced reactions are commonplace in biology. 
An example includes the creatine kinase shuttle, which enables the rapid exchange of nucleotides between the cell cytosol and mitochondrial matrix, where adenosine triphosphate (ATP) is hydrolyzed and synthesized, respectively \cite{vanBeek:2007gr}. 
This rapid exchange is in part dependent on the fast conductance of creatine through the mitochondrial outer membrane (MOM) by the voltage-dependent anion carrier (in a cation-selective mode) \cite{Anonymous:afLDblo3}, relative to hindered nucleotide transport.
%Am J Physiol. 1983 Nov;245(5 Pt 1):C423-7.
%Energy transport from mitochondria to myofibril by a creatine phosphate shuttle in cardiac cells.
%McClellan G, Weisberg A, Winegrad S. 
The MOM thus acts as a diffusional barrier to nucleotides and anions, giving rise to compartments of distinct nucleotide pools \cite{Anonymous:afLDblo3}; such compartmentalization can be shown to dampen fluctuations in the nucleotide concentration within the mitochondrium relative to the cytosol. % should be reported elsewhere, but using the default parameters for the van beek model, the ADP cyt ranges from 55-105, versus 20-50 in the IMS
Interestingly, a similar diffusional barrier has been proposed to exist directly adjacent to the cell membrane, which has been hypothesized to ensure a stable nucleotide supply to sarcolemmal ATPases \cite{Alekseev:2011wc}. 
Unlike the mitochondrial outer membrane that imposes an obvious diffusion barrier, the composition of the subsarcolemmal diffusional restriction is unknown. % check on wording in Alekseev artical
Thus, we investigated the hypothesis that a diffusional barrier arising from densely packed, charged
proteins is sufficient to support compartmentalized reactant pools with differing reaction dynamics.
We illustrated this for a simple equilibrium reaction (A $\leftrightarrow$ B) distributed across a
diffusional barrier, subject to a periodic source term acting on A within one compartment.
We then expanded this system by developing a model of an oscillatory reaction, whose periodicity arises due
to negative inhibition by a synthesized substrate, and demonstrated that the diffusional environment of the inhibitory substrate can tune the frequency of oscillations. 
These findings implicate the strong coupling of environment, spatial distribution and enzyme kinetics in shaping the dynamics of biochemical reactions. 


\subsection{Materials and methods}
\subsubsection{Homogenization}
(Put in JCP details)
We consider a lattice of globular proteins (of radius 13 Ang) that are evenly distributed. The spacing between proteins is varied such that the accessible volume fraction of of the lattice varies from 0.5 to 0.95. We further assume a net negative charge on the protein and a series of ligands whose unit charge varies between -1 and 1. Using our homogenization model \cite{KekenesHuskey:lolPRe3}(In review), we predict the normalized effective diffusion constants that represent the diffusional hindrance imposed by the charged lattice of proteins. 

\subsubsection{Finite element method} 
(Put in JCP details) 

\subsubsection{Compartments and Reactions} 
\lbi
\item We consider three compartments (1, 2 and 3) that are sequentially-linked. By altering the transport coefficients for compartment 1 into 2 and compartment 3 into 2 (and vice versa), we simulate the effect of a diffusional barrier between compartments 1 and 3, such that the substrate pools in compartment 1 exchange with compartment 3 by diffusion through compartment 2. 

\figshort{compartments.png}{compartments}{Reaction-diffusion compartments. Reactions are generally
restricted to compartments 1 and 3. Diffusion through compartment 2 may be fast (unrestricted) or slow
(hindered). } 


\item We consider the reaction, $A \leftrightarrow  B$, catalyzed by an enzyme E2, with the forward rate of
A into B is defined as kf and kb is the reverse rate. A second enzyme, E1, generates A in a sinusoidal
manner. Enzymes E1 and E2 are distributed in compartment 1, no enzymes are present in compartment 2 or 3.

\begin{align}
E1 \rightarrow A \\
E2 + A \leftrightarrow B
\end{align}

\item The Goodwin oscillator is represented by the reactions
\begin{verbatim}
Get eqns 1-3 from
\end{verbatim}

\begin{align}
E1 \rightarrow A \\
E2 + A \rightarrow B \\
E3 + B \rightarrow C \\
C -X\rightarrow E1
\end{align}


In our example, we restrict enzymes E1 and E2 into compartment 1 and enzyme E3 in compartment 3. As with the previous example, no reactions, only substrate diffusion occurs in compartment 2.

\lei

\subsubsection{ODE} 
(Put in Biophys details) 



\subsection{Results}
% Files are in compart.ipynb

\paragraph*{Homogenizaton predicts a range/upper bound of diffusivities for protein} 
In a previous study, we used homogenization theory to predict the effective diffusion constants for diffusion of a charged substrate (-1, 0, or 1) through a densely-packed lattice of charged globular proteins. The results in \fig{figshort:acsfig0} indicate that decreasing accessible volume fraction ($\phi$) tends to reduce the effective diffusion coefficient from its normalized bulk value of 1.0.  Repulsive interactions between substrate and the lattice yield the smallest diffusion constants, while attractive interactions yield the fastest constants. We utilize these effective diffusion constants in the reaction systems illustrated below. 
\figshort{acsfig0.png}{acsfig0}{Effective diffusion constant, D, for coions (red), counterions (blue) and neutral diffusers (black) for a unit cell with a 12.5 [\AA] negatively-charged, spherical protein. Attractive counte-
rion/protein interactions have faster diffusion relative to neutral and negatively charged diffusers.
Inclusion of attractive van der Waals interactions (+VDW) through the homogenized Smoluchowski equation with a DLVO poten-
tial increases the effective diffusion constants.}


\paragraph*{Diffusion barrier can depress fluctuations} 
%Need to use real parameters here. What do they mean? what is DA=DB=1?
In the figures below, we report the populations of A and B in each of the three compartments separated by diffusion barriers. In \fig{figshort:acsfig1a}, we consider the first reaction ($A \leftrightarrow B$), but use large diffusion constants that permit rapid diffusion across compartment 2, which is tantamount to the absence of a diffusion barrier. We observe that A and B approach equilibrium values of approximately 0.25 and 1.0, respectively. Furthermore, the fluctuation amplitudes of A due to the periodic source in compartment 1 are identical in all compartments. In contrast,  \fig{figshort:acsfig1b} we show distinct differences in compartments 1 and 2 when small diffusion constants are used. Namely, the fluctuations in compartment 3 are a small fraction of those observed in compartment 1. 
In \fig{figshort:acsfig1c}, we quantify the suppression of fluctuations in A in compartment 3 by the diffusional barrier (through varying the diffusion rate of A and B).  
Namely, we compare the covariance of A in compartments 1 and 3, normalized by the auto covariance of A in compartment 1. Values approaching zero correspond to a complete suppression of fluctuations in compartment 3, while values approaching 1 indicate that the fluctuations are identical in compartments 1 and 3. 
We observe that the fluctuations are strongly suppressed for diffusion constants less than 1, while suppression is minimized as D approaches a large rate corresponding to the absence of a diffusion barrier. 
\figshort{acsfig1a.png}{acsfig1a}{Concentrations of A (blue) and B (red) in compartments 1, 2 and three for the reaction $A \leftrightarrow  B$ in the absence of a diffusion barrier.} 
\figshort{acsfig1b.png}{acsfig1b}{Same as \fig{figshort:acsfig1a}, except in the presence of a diffusion barrier created by imposeing reduced diffusion constants into compartment 2.} 
\figshort{acsfig1c.png}{acsfig1c}{Degree of suppression of A fluctuations in compartment 3 by varying the diffusion barrier based on values from \fig{figshort:acsfig0}. Values decreasing from 1.0 indicate increasing degree of amplitude supression.} 

\paragraph*{Diffusion barrier can alter reaction frequency} 
In the figures below, we demonstrate how changes in the diffusion barrier imposed on a negative inhibitor impacts the frequency of a prototypical oscillatory model (Goodwin model). 
In \fig{figshort:acsfig2a} we demonstrate that the oscillator yields periodically-varying concentrations of species A (blue) and C (black,C is the inhibitor). In this case, the diffusion barriers are negligible given that a very large diffusion constant for C was used. \fig{figshort:acsfig2b}, we reduce the diffusion constants to 1.0 for species A and B, while we vary DC between 0.1 and 10.0. We observe that frequency of the A and C oscillations is largest when DC is large and decreases for smaller values of DC. \fig{figshort:acsfig2c} we compare the change in frequency over a range of diffusion constants and find that very small and very large diffusion constants can change the baseline frequency (1.0) by -80\% and +40\%, respectively. 
\figshort{acsfig2a.png}{acsfig2a}{Concentrations of A and C (the inhibitory substrate) for compartments 1-3 in the absence of a diffusion barrier. } 
\figshort{acsfig2b.png}{acsfig2b}{Same as \fig{figshort:acsfig2a}, but using $D_A=D_B=D_C=1.0$ (All slow), $D_A=D_B=1.0, \; D_C=0.1$ (slow, DC slow) and $D_A=D_B=1.0, \; D_C=10$ (slow, DC fast). } 
\figshort{acsfig2c.png}{acsfig2c}{Change C oscillation frequency with respect to $D_C=1.0$ by varying $D_C$ (mislabeled as Dz).} 


\subsection{Conclusions} 
A crowded cellular environment represented by a lattice of charged proteins can significantly vary the diffusion rate of substrates. 
We demonstrate that the dynamics of substrate pools can vary significantly between compartments that are separated by diffusional barriers typical of those suggested by our lattice model.
Furthermore, for oscillatory biochemical reactions, the presence of diffusion barriers can substantially shift baseline periods to lower frequencies. 
These findings provide insight into the ability of the cellular environment to tune biochemical reaction dynamics. 

%\subsection{Discussion}
%what was done
%
%limitation 
%
%for each finding
%	background?
%	why valid, how it jibes with other studies                     
%	consequences
%	
%very short conclusions
%	implications on field 
	
