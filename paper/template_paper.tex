Biophysical journal 


%Be sure to follow:
%\url{http://www.ccbb.pitt.edu/Faculty/zuckerman/ScienceWritingChecklist.pdf}
 “Multi-scale simulations of diffusion-influenced reactions”

\subsection{Abstract}
\lbi
\item what is the problem 
\item What was done
\item Key findings
\item Implications 
\lei

Kekenes-Huskey, PM; Eun, Changsun and JA McCammon

Biochemical reaction networks connect molecular signaling events with cellular function. 
A subset of these reactions are strongly influenced by diffusion ‘barriers’ arising from impenetrable cellular structures and macromolecules, as well as interactions between the reaction substrate, its environment and its target enzyme.
For these diffusion-influenced reactions, the spatial organization of diffusion barriers, specific and non-specific binding kinetics, and molecular composition may determine the timescale and amplitude of substrate signals. 
To this end we have developed simulation tools to determine how densely-packed diffusion obstacles, like proteins, and cellular structures give rise to  ’compartments’ within which substrate signals are decoupled, and how this decoupling depends on obstacle distribution and composition.
A key advance of this approach is the use of homogenization theory to coarse-grain hindrance of diffusion due to substrate/obstacle interactions.




\subsection{Introduction}
\lbi
\item INSERT MAJOR FIGURE FOR ENTIRE PAPER RIGHT HERE - write paper around this image 
\item General background and specifics on system of interest
\item Prior work that has been done
\item explanaton of what was done in paper 
\item key findings/spec aims
\lei

Biochemical reactions oftentimes stipulate diffusional resitrictions (Vendelin) 
Earlier showed how packing density of molecular structures can drastically reduce bulk diffusion constant. 
In this study, we seek to determine how diffusion barriers can influence reaction dynamics.



\subsection{Materials and methods}
\subsubsection{Homog}
Describe key homog paramers

\subsubsection{Finite element method} 
Describe key homog paramers

\subsubsection{Reactions} 
\lbi
\item compart
\item Goodwin 
\lei

\subsubsection{ODE} 


\subsection{Results}
%Lists all figures and tables (just describe data, don't explain)
% keep this list always, so I can reproduce data/figures easily
% \lbi
% \item expt: what it is doing
% \item files/locations/date:
% \item analysis: fig xxx 
% \lei

\paragraph*{Homogenizaton predicts a range/upper bound of diffusivities for protein} 
Fig what is it, explain colors. what they mean

\paragraph*{Diffusion barrier can depress fluctuations} 
Table - what is shows, how expressed
.....
\paragraph*{Diffusion barrier can alter reaction frequency} 


\subsection{Conclusions} 
Crowded environment can bound diffusion constants withincertain range 
Extent of crowding and crowderes can supress fluctuations betweebn domains 
Can act as shift frequency. 

%\subsection{Discussion}
%what was done
%
%limitation 
%
%for each finding
%	background?
%	why valid, how it jibes with other studies                     
%	consequences
%	
%very short conclusions
%	implications on field 
	
